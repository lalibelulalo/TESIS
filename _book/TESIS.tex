% Options for packages loaded elsewhere
\PassOptionsToPackage{unicode}{hyperref}
\PassOptionsToPackage{hyphens}{url}
%
\documentclass[
]{book}
\usepackage{amsmath,amssymb}
\usepackage{iftex}
\ifPDFTeX
  \usepackage[T1]{fontenc}
  \usepackage[utf8]{inputenc}
  \usepackage{textcomp} % provide euro and other symbols
\else % if luatex or xetex
  \usepackage{unicode-math} % this also loads fontspec
  \defaultfontfeatures{Scale=MatchLowercase}
  \defaultfontfeatures[\rmfamily]{Ligatures=TeX,Scale=1}
\fi
\usepackage{lmodern}
\ifPDFTeX\else
  % xetex/luatex font selection
\fi
% Use upquote if available, for straight quotes in verbatim environments
\IfFileExists{upquote.sty}{\usepackage{upquote}}{}
\IfFileExists{microtype.sty}{% use microtype if available
  \usepackage[]{microtype}
  \UseMicrotypeSet[protrusion]{basicmath} % disable protrusion for tt fonts
}{}
\makeatletter
\@ifundefined{KOMAClassName}{% if non-KOMA class
  \IfFileExists{parskip.sty}{%
    \usepackage{parskip}
  }{% else
    \setlength{\parindent}{0pt}
    \setlength{\parskip}{6pt plus 2pt minus 1pt}}
}{% if KOMA class
  \KOMAoptions{parskip=half}}
\makeatother
\usepackage{xcolor}
\usepackage{color}
\usepackage{fancyvrb}
\newcommand{\VerbBar}{|}
\newcommand{\VERB}{\Verb[commandchars=\\\{\}]}
\DefineVerbatimEnvironment{Highlighting}{Verbatim}{commandchars=\\\{\}}
% Add ',fontsize=\small' for more characters per line
\usepackage{framed}
\definecolor{shadecolor}{RGB}{248,248,248}
\newenvironment{Shaded}{\begin{snugshade}}{\end{snugshade}}
\newcommand{\AlertTok}[1]{\textcolor[rgb]{0.94,0.16,0.16}{#1}}
\newcommand{\AnnotationTok}[1]{\textcolor[rgb]{0.56,0.35,0.01}{\textbf{\textit{#1}}}}
\newcommand{\AttributeTok}[1]{\textcolor[rgb]{0.13,0.29,0.53}{#1}}
\newcommand{\BaseNTok}[1]{\textcolor[rgb]{0.00,0.00,0.81}{#1}}
\newcommand{\BuiltInTok}[1]{#1}
\newcommand{\CharTok}[1]{\textcolor[rgb]{0.31,0.60,0.02}{#1}}
\newcommand{\CommentTok}[1]{\textcolor[rgb]{0.56,0.35,0.01}{\textit{#1}}}
\newcommand{\CommentVarTok}[1]{\textcolor[rgb]{0.56,0.35,0.01}{\textbf{\textit{#1}}}}
\newcommand{\ConstantTok}[1]{\textcolor[rgb]{0.56,0.35,0.01}{#1}}
\newcommand{\ControlFlowTok}[1]{\textcolor[rgb]{0.13,0.29,0.53}{\textbf{#1}}}
\newcommand{\DataTypeTok}[1]{\textcolor[rgb]{0.13,0.29,0.53}{#1}}
\newcommand{\DecValTok}[1]{\textcolor[rgb]{0.00,0.00,0.81}{#1}}
\newcommand{\DocumentationTok}[1]{\textcolor[rgb]{0.56,0.35,0.01}{\textbf{\textit{#1}}}}
\newcommand{\ErrorTok}[1]{\textcolor[rgb]{0.64,0.00,0.00}{\textbf{#1}}}
\newcommand{\ExtensionTok}[1]{#1}
\newcommand{\FloatTok}[1]{\textcolor[rgb]{0.00,0.00,0.81}{#1}}
\newcommand{\FunctionTok}[1]{\textcolor[rgb]{0.13,0.29,0.53}{\textbf{#1}}}
\newcommand{\ImportTok}[1]{#1}
\newcommand{\InformationTok}[1]{\textcolor[rgb]{0.56,0.35,0.01}{\textbf{\textit{#1}}}}
\newcommand{\KeywordTok}[1]{\textcolor[rgb]{0.13,0.29,0.53}{\textbf{#1}}}
\newcommand{\NormalTok}[1]{#1}
\newcommand{\OperatorTok}[1]{\textcolor[rgb]{0.81,0.36,0.00}{\textbf{#1}}}
\newcommand{\OtherTok}[1]{\textcolor[rgb]{0.56,0.35,0.01}{#1}}
\newcommand{\PreprocessorTok}[1]{\textcolor[rgb]{0.56,0.35,0.01}{\textit{#1}}}
\newcommand{\RegionMarkerTok}[1]{#1}
\newcommand{\SpecialCharTok}[1]{\textcolor[rgb]{0.81,0.36,0.00}{\textbf{#1}}}
\newcommand{\SpecialStringTok}[1]{\textcolor[rgb]{0.31,0.60,0.02}{#1}}
\newcommand{\StringTok}[1]{\textcolor[rgb]{0.31,0.60,0.02}{#1}}
\newcommand{\VariableTok}[1]{\textcolor[rgb]{0.00,0.00,0.00}{#1}}
\newcommand{\VerbatimStringTok}[1]{\textcolor[rgb]{0.31,0.60,0.02}{#1}}
\newcommand{\WarningTok}[1]{\textcolor[rgb]{0.56,0.35,0.01}{\textbf{\textit{#1}}}}
\usepackage{longtable,booktabs,array}
\usepackage{calc} % for calculating minipage widths
% Correct order of tables after \paragraph or \subparagraph
\usepackage{etoolbox}
\makeatletter
\patchcmd\longtable{\par}{\if@noskipsec\mbox{}\fi\par}{}{}
\makeatother
% Allow footnotes in longtable head/foot
\IfFileExists{footnotehyper.sty}{\usepackage{footnotehyper}}{\usepackage{footnote}}
\makesavenoteenv{longtable}
\usepackage{graphicx}
\makeatletter
\def\maxwidth{\ifdim\Gin@nat@width>\linewidth\linewidth\else\Gin@nat@width\fi}
\def\maxheight{\ifdim\Gin@nat@height>\textheight\textheight\else\Gin@nat@height\fi}
\makeatother
% Scale images if necessary, so that they will not overflow the page
% margins by default, and it is still possible to overwrite the defaults
% using explicit options in \includegraphics[width, height, ...]{}
\setkeys{Gin}{width=\maxwidth,height=\maxheight,keepaspectratio}
% Set default figure placement to htbp
\makeatletter
\def\fps@figure{htbp}
\makeatother
\setlength{\emergencystretch}{3em} % prevent overfull lines
\providecommand{\tightlist}{%
  \setlength{\itemsep}{0pt}\setlength{\parskip}{0pt}}
\setcounter{secnumdepth}{5}
\usepackage{booktabs}
\ifLuaTeX
  \usepackage{selnolig}  % disable illegal ligatures
\fi
\usepackage[]{natbib}
\bibliographystyle{apalike}
\IfFileExists{bookmark.sty}{\usepackage{bookmark}}{\usepackage{hyperref}}
\IfFileExists{xurl.sty}{\usepackage{xurl}}{} % add URL line breaks if available
\urlstyle{same}
\hypersetup{
  pdftitle={A Minimal Book Example},
  pdfauthor={John Doe},
  hidelinks,
  pdfcreator={LaTeX via pandoc}}

\title{A Minimal Book Example}
\author{John Doe}
\date{2023-07-27}

\begin{document}
\maketitle

{
\setcounter{tocdepth}{1}
\tableofcontents
}
\hypertarget{about}{%
\chapter*{About}\label{about}}
\addcontentsline{toc}{chapter}{About}

This is a \emph{sample} book written in \textbf{Markdown}. You can use anything that Pandoc's Markdown supports; for example, a math equation \(a^2 + b^2 = c^2\).

\hypertarget{usage}{%
\section{Usage}\label{usage}}

Each \textbf{bookdown} chapter is an .Rmd file, and each .Rmd file can contain one (and only one) chapter. A chapter \emph{must} start with a first-level heading: \texttt{\#\ A\ good\ chapter}, and can contain one (and only one) first-level heading.

Use second-level and higher headings within chapters like: \texttt{\#\#\ A\ short\ section} or \texttt{\#\#\#\ An\ even\ shorter\ section}.

The \texttt{index.Rmd} file is required, and is also your first book chapter. It will be the homepage when you render the book.

\hypertarget{render-book}{%
\section{Render book}\label{render-book}}

You can render the HTML version of this example book without changing anything:

\begin{enumerate}
\def\labelenumi{\arabic{enumi}.}
\item
  Find the \textbf{Build} pane in the RStudio IDE, and
\item
  Click on \textbf{Build Book}, then select your output format, or select ``All formats'' if you'd like to use multiple formats from the same book source files.
\end{enumerate}

Or build the book from the R console:

\begin{Shaded}
\begin{Highlighting}[]
\NormalTok{bookdown}\SpecialCharTok{::}\FunctionTok{render\_book}\NormalTok{()}
\end{Highlighting}
\end{Shaded}

To render this example to PDF as a \texttt{bookdown::pdf\_book}, you'll need to install XeLaTeX. You are recommended to install TinyTeX (which includes XeLaTeX): \url{https://yihui.org/tinytex/}.

\hypertarget{preview-book}{%
\section{Preview book}\label{preview-book}}

As you work, you may start a local server to live preview this HTML book. This preview will update as you edit the book when you save individual .Rmd files. You can start the server in a work session by using the RStudio add-in ``Preview book'', or from the R console:

\begin{Shaded}
\begin{Highlighting}[]
\NormalTok{bookdown}\SpecialCharTok{::}\FunctionTok{serve\_book}\NormalTok{()}
\end{Highlighting}
\end{Shaded}

\hypertarget{introducciuxf3n}{%
\chapter{Introducción}\label{introducciuxf3n}}

El palíndromo altamente iterado 1 (HIP1 por sus siglas en inglés) cuya secuencia es 5'-GCGATCGC-3', está ampliamente representado en las cianobacterias con excepción de las pico-cianobacterias marinas y otros linajes. El origen de HIP1 y su función (si es que tiene alguna) permanecen desconocidos. Se ha observado que el sitio de reconocimiento (5'-Gm6ATC-3 `) de la enzima Dam metiltransferasa específica para adenina N6 de clase D12 (Dam-met) y el sitio de reconocimiento de DmtC (5'-m5CGATCG-3 ') están contenidos en HIP1, lo que sugiere una posible relación. Sin embargo, la asociación funcional de otros genes con HIP1 no se ha reportado. Informamos de un conjunto de genes cuyo patrón de presencia/ausencia está asociado con la abundancia de HIP1, lo que sugiere una relación funcional. También informamos de la presencia y representación excesiva de otros palíndromos altamente iterados (HIP) en linajes que carecen de HIP1. Para ello, utilizamos 127 genomas de cianobacterias que se clasificaron en dos grupos según la sobrerrepresentación de HIP1. Luego, nos preguntamos si hay familias génicas cuyo patrón de presencia/ausencia coincidiera con la clasificación de los genomas (con respecto a la sobrerrepresentación de HIP1). Ya que ninguna familia génica coincidió perfectamente, buscamos aquellas con al menos 80\% de coicidencia. 20 de 89 familias génicas identificadas, están asociadas con procesos que implican modificación y/o unión al ADN. Por ejemplo, aquellas con dominios que conforman transportadores ABC, de los que se ha sugerido una función como traslocadores de ADN en B. subtillis. También encontramos familias génicas que podrían participar en mecanismos anteriormente propuestos para aumentar la abundancia de HIP1. Por ejemplo, el mecanismo hipotético sugerido por Elhai J. (2015) en el cual participa el sistema de reparación dirigido por metilación (MMR). En relación con este último mecanismo identificamos un posible análogo de la proteína mutH específico para sitios metilados GmC y algunas enzimas con función de endonucleasa. Por ejemplo, algunas inteínas con dominio de endonucleasa como aquel que está presente en la ADN pol III sub unidad alfa, o como las proteínas con dominio GIY-GIY endonucleasa. El conjunto de familias génicas reportado en este trabajo podría servir para realizar experimentos posteriores para esclarecer la importancia de estas enzimas en el origen y evolución de HIP1.

\hypertarget{muxe9todos}{%
\chapter{Métodos}\label{muxe9todos}}

Se descargaron 2 conjuntos de genomas de cianobacterias del servidor del NCBI (\url{https://www.ncbi.nlm.nih.gov/genome/browse/\#!/prokaryotes}).

Estos conjuntos corresponden a:

\begin{itemize}
\tightlist
\item
  269 genomas completos y aquellos que solo contenian el cromosoma (complete\_chr)
\item
  165 genomas nuevos usados en Cabello-Yeves et al.~(2022) (pico)
\end{itemize}

Dichos genomas fueron descargados en formato Genebank (.gbk o gbff).

ESTE ES UN CAMBIO

  \bibliography{book.bib,packages.bib}

\end{document}
